\documentclass[12pt]{article}

\usepackage[style=authoryear]{biblatex}
\addbibresource{refs.bib}
%\bibliography{refs}
%\usepackage[toc,page]{appendix}

\usepackage[english]{babel}
\usepackage{csquotes}
\usepackage{caption}
\usepackage{float}
\usepackage{hyperref}
%\usepackage{lmodern}

\usepackage{amsmath,amsfonts,amssymb} % Math packages
\usepackage{amsthm}   % Theorems
\usepackage{array}    % Better tables
\usepackage{commath}  % derivatives and partials
\usepackage{enumitem} % List manipulation
\usepackage{float}    % Better positioning [H]
\usepackage{framed}   % Framed boxes
\usepackage{geometry} % Required for adjusting page dimensions and margins
\usepackage{graphicx} % Include images
\usepackage{multirow} % multicolumn tables
%\usepackage{pgfplots} % Create plots in latex
\usepackage{siunitx}  % SI unit system
\renewcommand{\arraystretch}{1.6}
%
%\pgfplotsset{compat=1.18}
%
\theoremstyle{definition}
\newtheorem*{definition}{Definition}

\newenvironment{define}[1]
	{\begin{framed}\begin{definition}{#1\\\\[1ex]}}
	{\end{definition}\end{framed}}


\geometry{
        paper=a4paper, % Paper size, change to letterpaper for US letter size
        top=2.5cm, % top margin
        bottom=2.5cm, % Bottom margin
        left=2.5cm, % left margin
        right=2.5cm, % Right margin
        headheight=14pt, % Header height
        footskip=1.5cm, % Space from the bottom margin to the baseline of the footer
        headsep=1.2cm, % Space from the top margin to the baseline of the header
}

\let\tss\textsuperscript % superscript macro
\let\oldtextbf\textbf
\renewcommand{\textbf}[1]{\oldtextbf{\boldmath #1}}

\newcommand{\reaction}[1]{\begin{equation}\ce{#1}\end{equation}}


% this change the content of the frontpage

%\ptype{Bachelor's Thesis}
\author{Luka Vest Büchmann}
\title{Optimizing Phylogenetic PCA}
%\subtitle{\textit{Subtitle TBD}}
%\advisor{Advisor: {Stefan Sommer}}
\date{\today}
%\fpimage{picture.png} % Remove this command if no image is desiredi





%\renewcommand{\contentsname}{Table of content}

\begin{document}

%\maketitle
%\newpage

\tableofcontents
\newpage

\section{Relevant Theory}


\subsection{The Spectral Theorem}
\begin{equation}
	S = Q \Sigma Q^T
	\label{eq:Spectral}
\end{equation}

\subsection{Singular Value Decomposition (SVD)}



\begin{equation}
	A = U \Sigma V^T
	\label{eq:SVD}
\end{equation}

\subsection{Principle Component Analysis}

From the previous section on SVD, we know that any matrix can be expressed as a product of an orthogonal, a diagonal, and another orthogonal matrix:
$$
A = U \Sigma V^T
$$ 
\textit{maybe explain how rest of sigmas/singvals are zero leading to only R pieces} \\
\textit{also not entirely clear myself on why} $\sigma_1 \ge \sigma_2 \ge \ldots \ge \sigma_R$

This can be further expressed as a sum of R rank-1 pieces, R being the original matrix A's rank.

$$
= \sigma_1 u_1 v_1^T + \ldots + \sigma_R u_R v_R^T
$$ 


These pieces are individually know as \textit{principal components}, and are, as the name suggests, key to PCA. 

Picking any $k \le R$, $A_k$ is defined as the sum of the first \textit{k} principal components: 
$$
A_k = \sigma_1 u_1 v_1^T + \ldots + \sigma_k u_k v_k^T
$$ 

We claim that this matrix $A_k$ is the closest possible approximation of $A$ with rank \textit{k}.
\ldots \textit{maybe explain norms and def. of ``approxomation'', add proofs for Eckart-Young}

This claim is neatly defined in the \textit{Eckart-Young Theorem}: 
\textit{italics?}
\begin{equation}
	\mathrm{rank}(B) = k \implies \|A - B\| \ge \|A - A_k\|
	\label{eq:Eckart-Young}
\end{equation}

\subsection{Phylogenetic PCA}



\cite{sommer}

\section{Demonstrative Work}




\printbibliography

\end{document}
